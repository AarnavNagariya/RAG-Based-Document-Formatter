\documentclass[conference]{IEEEtran}
\usepackage[utf8]{inputenc}
\usepackage{amsmath}
\usepackage{graphicx}
\usepackage{cite}
\usepackage{hyperref}

\title{VR-Phore Data Analysis}
\author{\IEEEauthorblockN{ }
\IEEEauthorblockA{\\
}
}

\begin{document}

\maketitle

\begin{abstract}
This paper analyzes data collected using the VR-Phore system, a virtual reality-based tool for measuring binocular alignment.  Unlike the synoptophore, VR-Phore employs a fixed reference image, allowing for precise quantification of the moving image's deviation.  The analysis focuses on interpreting displacement measurements (xDis, yDis, zDis, MagDis) and their corresponding degree of deviation, considering different slide types (fusion, simultaneous perception, stereopsis) and their impact on binocular alignment.  The first trial is excluded from analysis due to its familiarization purpose.
\end{abstract}

\begin{IEEEkeywords}
VR-Phore, Binocular Alignment, Synoptophore, Virtual Reality, Stereopsis, Fusion, Simultaneous Perception
\end{IEEEkeywords}

\section{Introduction}
This study investigates the use of VR-Phore for assessing binocular vision.  The VR-Phore methodology differs from traditional synoptophore methods by fixing one image and allowing movement of the other, enabling precise measurement of displacement and deviation. This paper presents and analyzes data from a single participant, examining the relationships between different slide types and the resulting displacement measurements.

\section{Methodology}
Data were collected using the VR-Phore system.  Participants adjusted the position of one image relative to a fixed reference image. Measurements recorded include x-axis displacement (\texttt{xDis}), y-axis displacement (\texttt{yDis}), z-axis displacement (\texttt{zDis}), total magnitude of displacement (\texttt{MagDis}), and the degree of deviation (\texttt{Degree}).  Different slide types were used, including fusion, simultaneous perception, and stereopsis slides.  Trial 1, a familiarization trial, was excluded from analysis.

\section{Results}
The following table presents the VR-Phore data for a single participant:

\begin{table}[h]
\centering
\begin{tabular}{|c|c|c|c|c|c|c|c|}
\hline
Testno & LeftMaterial & RightMaterial & Eye & xDis & yDis & zDis & MagDis & Degree \\
\hline
1 & BlackCat1 (Instance) & BlackCat2 (Instance) & Left & 0.092771 & 0.031805 & -0.00084 & 0.098075 & 1.123878 \\
2 & BlackCat2 (Instance) & BlackCat1 (Instance) & Right & 0.328212 & -0.06004 & 0.048567 & 0.337174 & 3.864466 \\
3 & BlackCat2 (Instance) & BlackCat1 (Instance) & Left & -0.00591 & -0.01714 & -0.35197 & 0.352437 & 4.039469 \\
4 & BlackCat2\_fove (Instance) & BlackCat1\_fove (Instance) & Right & 0.125868 & 0.08737 & 0.206699 & 0.257295 & 2.948712 \\
5 & BlackCat2\_fove (Instance) & BlackCat1\_fove (Instance) & Left & -0.02239 & 0.114287 & -0.22838 & 0.256356 & 2.937943 \\
6 & cross2 (Instance) & cross1 (Instance) & Right & 0.08976 & -0.05714 & -0.02129 & 0.108512 & 1.243478 \\
7 & cross2 (Instance) & cross1 (Instance) & Left & -0.0387 & 0.071799 & -0.34198 & 0.351576 & 4.029599 \\
8 & cross2\_fov (Instance) & cross1\_fov (Instance) & Right & 0.114046 & 0.042349 & 0.077183 & 0.144074 & 1.651018 \\
9 & cross2\_fov (Instance) & cross1\_fov (Instance) & Left & 0.068995 & -0.01573 & -0.25379 & 0.263474 & 3.019533 \\
10 & bucket2 (Instance) & bucket1 (Instance) & Right & 0.163759 & 0.042849 & 0.170084 & 0.239961 & 2.750019 \\
11 & bucket2 (Instance) & bucket1 (Instance) & Left & -0.49431 & -0.08756 & -0.39334 & 0.637745 & 7.312979 \\
12 & kit2 (Instance) & kit1 (Instance) & Right & 1.125022 & -2.51495 & 0.079412 & 2.756253 & 31.99856 \\
13 & kit2 (Instance) & kit1 (Instance) & Left & 0.939614 & -1.12586 & -0.42948 & 1.528034 & 17.57885 \\
\hline
\end{tabular}
\caption{VR-Phore Data for a Single Participant}
\label{tab:vrphore_data}
\end{table}

\section{Discussion}
The data in Table \ref{tab:vrphore_data} show varying degrees of displacement and deviation across different slide types.  Tests 2-5 (fusion slides) exhibit lower deviation than tests 6-9 (simultaneous perception) and tests 10-13 (stereopsis slides), suggesting potential differences in binocular alignment challenges presented by these slide types.  The presence of z-axis displacement, despite 2D manipulation, is attributed to the cylindrical image movement within the virtual environment. Further analysis is needed to determine if this pattern is consistent across participants.

\section{Conclusion}
This preliminary analysis of VR-Phore data demonstrates its potential for quantifying binocular alignment. The observed variations in displacement and deviation across different slide types highlight the system's sensitivity to various aspects of binocular vision. Future work will involve larger sample sizes and more detailed statistical analyses to validate these findings.


\section*{Acknowledgments}
\bibliographystyle{IEEEtran}
\bibliography{references}

\end{document}
